% Options for packages loaded elsewhere
\PassOptionsToPackage{unicode}{hyperref}
\PassOptionsToPackage{hyphens}{url}
\PassOptionsToPackage{dvipsnames,svgnames*,x11names*}{xcolor}
%
\documentclass[
  12pt]{findlay}
\usepackage{lmodern}
\usepackage{amssymb,amsmath}
\usepackage{ifxetex,ifluatex}
\ifnum 0\ifxetex 1\fi\ifluatex 1\fi=0 % if pdftex
  \usepackage[T1]{fontenc}
  \usepackage[utf8]{inputenc}
  \usepackage{textcomp} % provide euro and other symbols
\else % if luatex or xetex
  \usepackage{unicode-math}
  \defaultfontfeatures{Scale=MatchLowercase}
  \defaultfontfeatures[\rmfamily]{Ligatures=TeX,Scale=1}
\fi
% Use upquote if available, for straight quotes in verbatim environments
\IfFileExists{upquote.sty}{\usepackage{upquote}}{}
\IfFileExists{microtype.sty}{% use microtype if available
  \usepackage[]{microtype}
  \UseMicrotypeSet[protrusion]{basicmath} % disable protrusion for tt fonts
}{}
\makeatletter
\@ifundefined{KOMAClassName}{% if non-KOMA class
  \IfFileExists{parskip.sty}{%
    \usepackage{parskip}
  }{% else
    \setlength{\parindent}{0pt}
    \setlength{\parskip}{6pt plus 2pt minus 1pt}}
}{% if KOMA class
  \KOMAoptions{parskip=half}}
\makeatother
\usepackage{xcolor}
\IfFileExists{xurl.sty}{\usepackage{xurl}}{} % add URL line breaks if available
\IfFileExists{bookmark.sty}{\usepackage{bookmark}}{\usepackage{hyperref}}
\hypersetup{
  colorlinks=true,
  linkcolor=black,
  filecolor=Maroon,
  citecolor=Green,
  urlcolor=blue,
  pdfcreator={LaTeX via pandoc}}
\urlstyle{same} % disable monospaced font for URLs
\usepackage{listings}
\newcommand{\passthrough}[1]{#1}
\lstset{defaultdialect=[5.3]Lua}
\lstset{defaultdialect=[x86masm]Assembler}
\setlength{\emergencystretch}{3em} % prevent overfull lines
\providecommand{\tightlist}{%
  \setlength{\itemsep}{0pt}\setlength{\parskip}{0pt}}
\setcounter{secnumdepth}{5}
\usepackage[]{biblatex}

\title{COMP5900 OS Security}
\date{\today}

\begin{document}
\maketitle

\hypertarget{introduction}{%
\section{Introduction}\label{introduction}}

\begin{itemize}
\tightlist
\item
  trusted computing base

  \begin{itemize}
  \tightlist
  \item
    applications that are essential to functioning of the OS
  \item
    e.g., passwd
  \item
    these would probably be okay to talk about for OS vuln. but
    kernelspace code is preferred
  \end{itemize}
\end{itemize}

\hypertarget{blooms-taxonomy}{%
\subsection{Bloom's Taxonomy}\label{blooms-taxonomy}}

\begin{itemize}
\tightlist
\item
  course targets top 3 sections (for evaluation)

  \begin{itemize}
  \tightlist
  \item
    create
  \item
    evaluate
  \item
    analyze
  \item
    (some understanding)
  \end{itemize}
\end{itemize}

\hypertarget{what-is-an-os}{%
\subsection{What is an OS?}\label{what-is-an-os}}

\begin{itemize}
\tightlist
\item
  kernel
\item
  essential applications (systemd, passwd, etc.)
\item
  what does it do?

  \begin{itemize}
  \tightlist
  \item
    scheduling
  \item
    network stack
  \item
    file systems
  \item
    block I/O on disk
  \item
    hardware interrupts (e.g.~I/O)
  \item
    at least basic access control (memory protection, etc.)
  \item
    often runs in supervisor mode (apparently not necessarily? But I
    don't agree with this\ldots)
  \end{itemize}
\end{itemize}

\hypertarget{what-is-the-most-secure-os}{%
\subsection{What is the Most Secure
OS?}\label{what-is-the-most-secure-os}}

\begin{itemize}
\tightlist
\item
  probably something task-specific
\end{itemize}

\hypertarget{group-activity-come-up-with-an-os-less-implementation-for-a-word-processor}{%
\subsection{Group Activity: Come up with an OS-less implementation for a
word
processor}\label{group-activity-come-up-with-an-os-less-implementation-for-a-word-processor}}

\begin{itemize}
\tightlist
\item
  interrupt handling for keyboard

  \begin{itemize}
  \tightlist
  \item
    need at least some basic scheduler that can pause and resume main
    execution
  \end{itemize}
\item
  interface with monitor for graphical display
\item
  block I/O driver for disk

  \begin{itemize}
  \tightlist
  \item
    filesystem to organize data
  \end{itemize}
\item
  hmm\ldots{} this is starting to feel like we just implemented our own
  task-specific OS

  \begin{itemize}
  \tightlist
  \item
    that's the main takeaway here!
  \end{itemize}
\end{itemize}

\hypertarget{secure-os}{%
\section{Secure OS}\label{secure-os}}

\hypertarget{van-oorschot-chapter-5.0---5.2}{%
\subsection{Van Oorschot Chapter 5.0 -
5.2}\label{van-oorschot-chapter-5.0---5.2}}

\hypertarget{intro}{%
\subsubsection{Intro}\label{intro}}

\begin{itemize}
\tightlist
\item
  early security had same challenges we face today

  \begin{itemize}
  \tightlist
  \item
    protecting programs from others
  \item
    restricting access to resources
  \item
    ``protection'' means mostly memory access control
  \end{itemize}
\item
  memory is important

  \begin{itemize}
  \tightlist
  \item
    holds data
  \item
    holds programs
  \item
    I/O devices through memory address and files
  \item
    files -\textgreater{} both main memory and secondary storage
  \end{itemize}
\item
  early protection

  \begin{itemize}
  \tightlist
  \item
    virtual addresses
  \item
    access control lists
  \item
    limited process address space
  \item
    these fundamentals are still used today
  \end{itemize}
\item
  Multics

  \begin{itemize}
  \tightlist
  \item
    security very influential in its early design
  \item
    original UNIX was heavily based on Multics
  \end{itemize}
\end{itemize}

\hypertarget{memory-protection-supervisor-mode-and-accountability}{%
\subsubsection{Memory protection, supervisor mode, and
accountability}\label{memory-protection-supervisor-mode-and-accountability}}

\begin{itemize}
\tightlist
\item
  batch processing

  \begin{itemize}
  \tightlist
  \item
    prepare jobs ahead of time and submit them together as a batch job
  \end{itemize}
\item
  time-sharing systems

  \begin{itemize}
  \tightlist
  \item
    allowed shared use of a single computer
  \item
    (preferable to batch jobs from a usability standpoint)
  \item
    same way single-user computers work today with one user running many
    programs
  \end{itemize}
\item
  resource conflicts

  \begin{itemize}
  \tightlist
  \item
    processes running simultaneously can try to access the same
    resources
  \item
    intentionally or otherwise
  \item
    if a program could access full memory of the machine, errors could
    corrupt OS data or code
  \end{itemize}
\item
  supervisor

  \begin{itemize}
  \tightlist
  \item
    runs with higher permissions in the protection CPU (ring 0, 1, 2)
  \item
    no other program can alter the privileged bit
  \item
    a special machine instruction immediately transfers control to the
    supervisor
  \end{itemize}
\item
  privileged bit

  \begin{itemize}
  \tightlist
  \item
    process is running in supervisor mode
  \end{itemize}
\item
  descriptor register

  \begin{itemize}
  \tightlist
  \item
    holds a memory descriptor that describes base and upper bound
  \item
    lowest addressable memory by a process and a number of words from
    that point that are addressable
  \end{itemize}
\item
  limitations of memory-range based protection

  \begin{itemize}
  \tightlist
  \item
    all-or-nothing mode of control
  \item
    either you have full access or no access
  \item
    allows full isolation, but not fine-grained sharing
  \end{itemize}
\item
  segment addressing with access permissions

  \begin{itemize}
  \tightlist
  \item
    segment = collection of words representing a logical unit of
    information
  \item
    descriptor segment per process maintained by OS

    \begin{itemize}
    \tightlist
    \item
      holds segment descriptors that define addressable memory and
      permissions
    \end{itemize}
  \item
    descriptor base register points to memory descriptor of active
    process
  \end{itemize}
\item
  permissions on virtual segments

  \begin{itemize}
  \tightlist
  \item
    R non-supervisor can read
  \item
    W can be written to
  \item
    X can be executed
  \item
    M run in supervisor mode (if X)
  \item
    F all access attempts trap to supervisor
  \item
    now the same physical segment can be given different access for
    different processes
  \end{itemize}
\item
  accountability, UIDs, and principals

  \begin{itemize}
  \tightlist
  \item
    UID (maps users to a unique identifier)
  \item
    ``principal'' -\textgreater{} abstracts the entity responsible for
    code execution from the actual user or program actions
  \item
    UID is the primary basis for granting permissions
  \end{itemize}
\item
  roles

  \begin{itemize}
  \tightlist
  \item
    assign distinct UIDs to distinct privileges
  \item
    should follow principle of least privilege
  \end{itemize}
\end{itemize}

\hypertarget{reference-monitor-access-matrix-security-kernel}{%
\subsubsection{Reference monitor, access matrix, security
kernel}\label{reference-monitor-access-matrix-security-kernel}}

\begin{itemize}
\tightlist
\item
  reference monitor

  \begin{itemize}
  \tightlist
  \item
    concept that ``all references by any program to any other program,
    data, or device are validated''
  \item
    conceptualized as one reference monitor, but in practice, would be a
    lot of reference monitors working together
  \end{itemize}
\item
  access matrix

  \begin{itemize}
  \tightlist
  \item
    2D matrix of subjects, objects
  \item
    taking a row (subject) gives a capabilities list
  \item
    taking a column (object) gives an access control list
  \item
    each intersection in this matrix defines a set of permissions
  \end{itemize}
\item
  security kernel

  \begin{itemize}
  \tightlist
  \item
    reference validation
  \item
    audit trails via audit logs (user X did Y at time Z)

    \begin{itemize}
    \tightlist
    \item
      these might not necessarily need to be tamper-proof, depends on
      needs
    \end{itemize}
  \item
    needs to be:

    \begin{itemize}
    \tightlist
    \item
      tamper-proof
    \item
      always invoked (not circumventable)
    \item
      verifiable (needs to be minimal / small enough to make this
      possible )
    \end{itemize}
  \end{itemize}
\item
  protection mechanisms

  \begin{itemize}
  \tightlist
  \item
    ticket-oriented (capabilities)

    \begin{itemize}
    \tightlist
    \item
      access token allows entry to an event, as long as ticket is
      authentic
    \end{itemize}
  \item
    id-based

    \begin{itemize}
    \tightlist
    \item
      authorization lists based on ID
    \end{itemize}
  \end{itemize}
\end{itemize}

\hypertarget{jaeger-chapter-1}{%
\subsection{Jaeger Chapter 1}\label{jaeger-chapter-1}}

\hypertarget{secure-os-1}{%
\subsubsection{Secure OS}\label{secure-os-1}}

\hypertarget{section}{%
\subsubsection{}\label{section}}

\hypertarget{jaeger-chapter-2}{%
\subsection{Jaeger Chapter 2}\label{jaeger-chapter-2}}

\printbibliography

\end{document}
