% Options for packages loaded elsewhere
\PassOptionsToPackage{unicode}{hyperref}
\PassOptionsToPackage{hyphens}{url}
\PassOptionsToPackage{dvipsnames,svgnames*,x11names*}{xcolor}
%
\documentclass[
  12pt]{findlay}
\usepackage{lmodern}
\usepackage{amssymb,amsmath}
\usepackage{ifxetex,ifluatex}
\ifnum 0\ifxetex 1\fi\ifluatex 1\fi=0 % if pdftex
  \usepackage[T1]{fontenc}
  \usepackage[utf8]{inputenc}
  \usepackage{textcomp} % provide euro and other symbols
\else % if luatex or xetex
  \usepackage{unicode-math}
  \defaultfontfeatures{Scale=MatchLowercase}
  \defaultfontfeatures[\rmfamily]{Ligatures=TeX,Scale=1}
\fi
% Use upquote if available, for straight quotes in verbatim environments
\IfFileExists{upquote.sty}{\usepackage{upquote}}{}
\IfFileExists{microtype.sty}{% use microtype if available
  \usepackage[]{microtype}
  \UseMicrotypeSet[protrusion]{basicmath} % disable protrusion for tt fonts
}{}
\makeatletter
\@ifundefined{KOMAClassName}{% if non-KOMA class
  \IfFileExists{parskip.sty}{%
    \usepackage{parskip}
  }{% else
    \setlength{\parindent}{0pt}
    \setlength{\parskip}{6pt plus 2pt minus 1pt}}
}{% if KOMA class
  \KOMAoptions{parskip=half}}
\makeatother
\usepackage{xcolor}
\IfFileExists{xurl.sty}{\usepackage{xurl}}{} % add URL line breaks if available
\IfFileExists{bookmark.sty}{\usepackage{bookmark}}{\usepackage{hyperref}}
\hypersetup{
  colorlinks=true,
  linkcolor=black,
  filecolor=Maroon,
  citecolor=Green,
  urlcolor=blue,
  pdfcreator={LaTeX via pandoc}}
\urlstyle{same} % disable monospaced font for URLs
\usepackage{listings}
\newcommand{\passthrough}[1]{#1}
\lstset{defaultdialect=[5.3]Lua}
\lstset{defaultdialect=[x86masm]Assembler}
\setlength{\emergencystretch}{3em} % prevent overfull lines
\providecommand{\tightlist}{%
  \setlength{\itemsep}{0pt}\setlength{\parskip}{0pt}}
\setcounter{secnumdepth}{5}
\usepackage[T1]{fontenc}
\usepackage{cmap}
\input{glyphtounicode}
\pdfgentounicode=1
\usepackage[]{biblatex}

\title{COMP5900 OS Security}
\date{\today}

\begin{document}
\maketitle

\hypertarget{introduction}{%
\section{Introduction}\label{introduction}}

\begin{itemize}
\tightlist
\item
  trusted computing base

  \begin{itemize}
  \tightlist
  \item
    applications that are essential to functioning of the OS
  \item
    e.g., passwd
  \item
    these would probably be okay to talk about for OS vuln. but
    kernelspace code is preferred
  \end{itemize}
\end{itemize}

\hypertarget{blooms-taxonomy}{%
\subsection{Bloom's Taxonomy}\label{blooms-taxonomy}}

\begin{itemize}
\tightlist
\item
  course targets top 3 sections (for evaluation)

  \begin{itemize}
  \tightlist
  \item
    create
  \item
    evaluate
  \item
    analyze
  \item
    (some understanding)
  \end{itemize}
\end{itemize}

\hypertarget{what-is-an-os}{%
\subsection{What is an OS?}\label{what-is-an-os}}

\begin{itemize}
\tightlist
\item
  kernel
\item
  essential applications (systemd, passwd, etc.)
\item
  what does it do?

  \begin{itemize}
  \tightlist
  \item
    scheduling
  \item
    network stack
  \item
    file systems
  \item
    block I/O on disk
  \item
    hardware interrupts (e.g.~I/O)
  \item
    at least basic access control (memory protection, etc.)
  \item
    often runs in supervisor mode (apparently not necessarily? But I
    don't agree with this\ldots)
  \end{itemize}
\end{itemize}

\hypertarget{what-is-the-most-secure-os}{%
\subsection{What is the Most Secure
OS?}\label{what-is-the-most-secure-os}}

\begin{itemize}
\tightlist
\item
  probably something task-specific
\end{itemize}

\hypertarget{group-activity-come-up-with-an-os-less-implementation-for-a-word-processor}{%
\subsection{Group Activity: Come up with an OS-less implementation for a
word
processor}\label{group-activity-come-up-with-an-os-less-implementation-for-a-word-processor}}

\begin{itemize}
\tightlist
\item
  interrupt handling for keyboard

  \begin{itemize}
  \tightlist
  \item
    need at least some basic scheduler that can pause and resume main
    execution
  \end{itemize}
\item
  interface with monitor for graphical display
\item
  block I/O driver for disk

  \begin{itemize}
  \tightlist
  \item
    filesystem to organize data
  \end{itemize}
\item
  hmm\ldots{} this is starting to feel like we just implemented our own
  task-specific OS

  \begin{itemize}
  \tightlist
  \item
    that's the main takeaway here!
  \end{itemize}
\end{itemize}

\hypertarget{secure-os}{%
\section{Secure OS}\label{secure-os}}

\hypertarget{van-oorschot-chapter-5.0---5.2}{%
\subsection{Van Oorschot Chapter 5.0 -
5.2}\label{van-oorschot-chapter-5.0---5.2}}

\hypertarget{intro}{%
\subsubsection{Intro}\label{intro}}

\begin{itemize}
\tightlist
\item
  early security had same challenges we face today

  \begin{itemize}
  \tightlist
  \item
    protecting programs from others
  \item
    restricting access to resources
  \item
    ``protection'' means mostly memory access control
  \end{itemize}
\item
  memory is important

  \begin{itemize}
  \tightlist
  \item
    holds data
  \item
    holds programs
  \item
    I/O devices through memory address and files
  \item
    files -\textgreater{} both main memory and secondary storage
  \end{itemize}
\item
  early protection

  \begin{itemize}
  \tightlist
  \item
    virtual addresses
  \item
    access control lists
  \item
    limited process address space
  \item
    these fundamentals are still used today
  \end{itemize}
\item
  Multics

  \begin{itemize}
  \tightlist
  \item
    security very influential in its early design
  \item
    original UNIX was heavily based on Multics
  \end{itemize}
\end{itemize}

\hypertarget{memory-protection-supervisor-mode-and-accountability}{%
\subsubsection{Memory protection, supervisor mode, and
accountability}\label{memory-protection-supervisor-mode-and-accountability}}

\begin{itemize}
\tightlist
\item
  batch processing

  \begin{itemize}
  \tightlist
  \item
    prepare jobs ahead of time and submit them together as a batch job
  \end{itemize}
\item
  time-sharing systems

  \begin{itemize}
  \tightlist
  \item
    allowed shared use of a single computer
  \item
    (preferable to batch jobs from a usability standpoint)
  \item
    same way single-user computers work today with one user running many
    programs
  \end{itemize}
\item
  resource conflicts

  \begin{itemize}
  \tightlist
  \item
    processes running simultaneously can try to access the same
    resources
  \item
    intentionally or otherwise
  \item
    if a program could access full memory of the machine, errors could
    corrupt OS data or code
  \end{itemize}
\item
  supervisor

  \begin{itemize}
  \tightlist
  \item
    runs with higher permissions in the protection CPU (ring 0, 1, 2)
  \item
    no other program can alter the privileged bit
  \item
    a special machine instruction immediately transfers control to the
    supervisor
  \end{itemize}
\item
  privileged bit

  \begin{itemize}
  \tightlist
  \item
    process is running in supervisor mode
  \end{itemize}
\item
  descriptor register

  \begin{itemize}
  \tightlist
  \item
    holds a memory descriptor that describes base and upper bound
  \item
    lowest addressable memory by a process and a number of words from
    that point that are addressable
  \end{itemize}
\item
  limitations of memory-range based protection

  \begin{itemize}
  \tightlist
  \item
    all-or-nothing mode of control
  \item
    either you have full access or no access
  \item
    allows full isolation, but not fine-grained sharing
  \end{itemize}
\item
  segment addressing with access permissions

  \begin{itemize}
  \tightlist
  \item
    segment = collection of words representing a logical unit of
    information
  \item
    descriptor segment per process maintained by OS

    \begin{itemize}
    \tightlist
    \item
      holds segment descriptors that define addressable memory and
      permissions
    \end{itemize}
  \item
    descriptor base register points to memory descriptor of active
    process
  \end{itemize}
\item
  permissions on virtual segments

  \begin{itemize}
  \tightlist
  \item
    R non-supervisor can read
  \item
    W can be written to
  \item
    X can be executed
  \item
    M run in supervisor mode (if X)
  \item
    F all access attempts trap to supervisor
  \item
    now the same physical segment can be given different access for
    different processes
  \end{itemize}
\item
  accountability, UIDs, and principals

  \begin{itemize}
  \tightlist
  \item
    UID (maps users to a unique identifier)
  \item
    ``principal'' -\textgreater{} abstracts the entity responsible for
    code execution from the actual user or program actions
  \item
    UID is the primary basis for granting permissions
  \end{itemize}
\item
  roles

  \begin{itemize}
  \tightlist
  \item
    assign distinct UIDs to distinct privileges
  \item
    should follow principle of least privilege
  \end{itemize}
\end{itemize}

\hypertarget{reference-monitor-access-matrix-security-kernel}{%
\subsubsection{Reference monitor, access matrix, security
kernel}\label{reference-monitor-access-matrix-security-kernel}}

\begin{itemize}
\tightlist
\item
  reference monitor

  \begin{itemize}
  \tightlist
  \item
    concept that ``all references by any program to any other program,
    data, or device are validated''
  \item
    conceptualized as one reference monitor, but in practice, would be a
    lot of reference monitors working together
  \end{itemize}
\item
  access matrix

  \begin{itemize}
  \tightlist
  \item
    2D matrix of subjects, objects
  \item
    taking a row (subject) gives a capabilities list
  \item
    taking a column (object) gives an access control list
  \item
    each intersection in this matrix defines a set of permissions
  \end{itemize}
\item
  security kernel

  \begin{itemize}
  \tightlist
  \item
    reference validation
  \item
    audit trails via audit logs (user X did Y at time Z)

    \begin{itemize}
    \tightlist
    \item
      these might not necessarily need to be tamper-proof, depends on
      needs
    \end{itemize}
  \item
    needs to be:

    \begin{itemize}
    \tightlist
    \item
      tamper-proof
    \item
      always invoked (not circumventable)
    \item
      verifiable (needs to be minimal / small enough to make this
      possible )
    \end{itemize}
  \end{itemize}
\item
  protection mechanisms

  \begin{itemize}
  \tightlist
  \item
    ticket-oriented (capabilities)

    \begin{itemize}
    \tightlist
    \item
      access token allows entry to an event, as long as ticket is
      authentic
    \end{itemize}
  \item
    id-based

    \begin{itemize}
    \tightlist
    \item
      authorization lists based on ID
    \end{itemize}
  \end{itemize}
\end{itemize}

\hypertarget{jaeger-chapter-1}{%
\subsection{Jaeger Chapter 1}\label{jaeger-chapter-1}}

\begin{itemize}
\tightlist
\item
  general-purpose -\textgreater{} complex
\item
  task-specific -\textgreater{} not so complex
\item
  general purpose OS are hard to secure because of their complexity
\item
  ensuring security depends on securing

  \begin{itemize}
  \tightlist
  \item
    resource mechanisms
  \item
    scheduling mechanisms
  \end{itemize}
\end{itemize}

\hypertarget{secure-os-1}{%
\subsubsection{Secure OS}\label{secure-os-1}}

\begin{itemize}
\tightlist
\item
  enforce security goals despite the threats faced by the system

  \begin{itemize}
  \tightlist
  \item
    implement security mechanisms to do this
  \end{itemize}
\item
  secure OS possible?

  \begin{itemize}
  \tightlist
  \item
    probably not
  \item
    a modern OS by definition can probably never be 100\% secure
  \item
    security as a negative goal
  \end{itemize}
\item
  understanding secure OS requires understanding

  \begin{itemize}
  \tightlist
  \item
    security goals
  \item
    trust model
  \item
    threat model
  \end{itemize}
\end{itemize}

\hypertarget{security-goals}{%
\subsubsection{Security Goals}\label{security-goals}}

\begin{itemize}
\tightlist
\item
  define operations that can be executed by a system while remaining in
  a secure state

  \begin{itemize}
  \tightlist
  \item
    i.e.~prevent unauthorized access
  \end{itemize}
\item
  high level of abstraction
\item
  define a requirement that the system's design can then satisfy
\item
  we want to maintain: secrecy, integrity, availability

  \begin{itemize}
  \tightlist
  \item
    secrecy = limit read access for objects by subjects
  \item
    integrity = limit the write access for objects by subjects
  \item
    availability = limit the resources that a subject may consume
    (i.e.~no DoS)
  \end{itemize}
\item
  subjects

  \begin{itemize}
  \tightlist
  \item
    users, processes, etc.
  \end{itemize}
\item
  objects

  \begin{itemize}
  \tightlist
  \item
    resources of the system that subjects may or may not access in
    various ways
  \item
    e.g.~files, sockets, memory
  \end{itemize}
\item
  security goals can be

  \begin{itemize}
  \tightlist
  \item
    defined by function (e.g.~principle of least privilege)
  \item
    defined by requirements (e.g.~simple-security property)
  \end{itemize}
\end{itemize}

\hypertarget{trust-model}{%
\subsubsection{Trust Model}\label{trust-model}}

\begin{itemize}
\tightlist
\item
  trust model

  \begin{itemize}
  \tightlist
  \item
    defines the set of software and data we trust to help us enforce our
    security goals
  \item
    we depend on this model to correctly enforce our security goals
  \end{itemize}
\item
  trusted computing base

  \begin{itemize}
  \tightlist
  \item
    trust model for an operating system
  \end{itemize}
\item
  TCB should \textbf{ideally} be minimal to the extent that we require

  \begin{itemize}
  \tightlist
  \item
    in practice, this is a wide variety of software
  \end{itemize}
\item
  TCB includes

  \begin{itemize}
  \tightlist
  \item
    all OS code (assuming no boundaries as in a monolithic kernel)
  \item
    other software that defines our security goals
  \item
    other software that enforces our security goals
  \item
    software that bootstraps the above
  \item
    software like Xorg that performs actions on behalf of all other
    processes
  \end{itemize}
\item
  a secure OS developer needs to prove their system has a viable trust
  model

  \begin{enumerate}
  \def\labelenumi{(\arabic{enumi})}
  \tightlist
  \item
    TCB must mediate all sensitive operations
  \item
    verification of the TCB software and data
  \item
    verification of TCB tamper-resistance
  \end{enumerate}
\item
  identifying and verifying TCB is a complex and non-trivial task
\end{itemize}

\hypertarget{threat-model}{%
\subsubsection{Threat Model}\label{threat-model}}

\begin{itemize}
\tightlist
\item
  defines a set of operations that an attacker may use to compromise the
  system
\item
  assume a powerful attacker who

  \begin{itemize}
  \tightlist
  \item
    can inject operations from the network
  \item
    may be in control of non-TCB applications
  \end{itemize}
\item
  if the attacker finds a vulnerability that violates secrecy or
  integrity goals, the system is compromised
\item
  highlights a critical weakness in commercial OSes

  \begin{itemize}
  \tightlist
  \item
    assume that all software running on behalf of a subject is trusted
    by the subject
  \end{itemize}
\item
  our task? protect the TCB from threats

  \begin{itemize}
  \tightlist
  \item
    easier said than done
  \item
    user interacts with a variety of processes
  \item
    users are untrusted
  \item
    TCB interacts with a variety of untrusted processes
  \end{itemize}
\end{itemize}

\hypertarget{jaeger-chapter-2}{%
\subsection{Jaeger Chapter 2}\label{jaeger-chapter-2}}

\hypertarget{protection-system}{%
\subsubsection{Protection System}\label{protection-system}}

\begin{itemize}
\tightlist
\item
  protection system consists of

  \begin{itemize}
  \tightlist
  \item
    protection state
  \item
    protection state operations
  \end{itemize}
\item
  protection state

  \begin{itemize}
  \tightlist
  \item
    what operations can subjects perform on objects
  \end{itemize}
\item
  protection state operations

  \begin{itemize}
  \tightlist
  \item
    what operations can modify the protections state
  \item
    (this is distinct from the operations that the protection state
    describes)
  \end{itemize}
\end{itemize}

\hypertarget{lampsons-access-matrix}{%
\paragraph{Lampson's Access Matrix}\label{lampsons-access-matrix}}

\begin{itemize}
\tightlist
\item
  protection state

  \begin{itemize}
  \tightlist
  \item
    rows = subjects
  \item
    cols = objects
  \item
    select row -\textgreater{} capability list
  \item
    select col -\textgreater{} access control list
  \item
    each entry specified privileges subject -\textgreater{} object
  \end{itemize}
\item
  protection state operations

  \begin{itemize}
  \tightlist
  \item
    determine which processes can modify cells
  \end{itemize}
\end{itemize}

\hypertarget{mandatory-protection-systems}{%
\paragraph{Mandatory Protection
Systems}\label{mandatory-protection-systems}}

\begin{itemize}
\tightlist
\item
  we don't want untrusted processes tampering with the protection
  system's state by adding subjects, objects, operations
\item
  discretionary access control system (DAC)

  \begin{itemize}
  \tightlist
  \item
    an access control system that permits untrusted modification
  \item
    \emph{safety problem}

    \begin{itemize}
    \tightlist
    \item
      how do we ensure that all possible states deriving from initial
      state will not provide unauthorized access
    \end{itemize}
  \end{itemize}
\item
  mandatory protection systems / mandatory access control (MAC)

  \begin{itemize}
  \tightlist
  \item
    protection system can only be modified by trusted administrators via
    trusted software
  \item
    mandatory protection state -\textgreater{} subjects and objects are
    represented by labels

    \begin{itemize}
    \tightlist
    \item
      state describes operations subject labels -\textgreater{} object
      labels
    \end{itemize}
  \item
    labeling state

    \begin{itemize}
    \tightlist
    \item
      state for mapping subjects and objects to labels
    \end{itemize}
  \item
    transition state

    \begin{itemize}
    \tightlist
    \item
      describes legal ways subjects and objects may be relabeled
    \end{itemize}
  \end{itemize}
\item
  set of labels being fixed in MAC doesn't mean that set of
  subjects/objects are fixed

  \begin{itemize}
  \tightlist
  \item
    we can dynamically assign labels to created subjects and objects
    (labeling state)
  \item
    we can dynamically relabel subjects and objects/resources
    (transition state)
  \end{itemize}
\end{itemize}

\hypertarget{reference-monitor}{%
\subsubsection{Reference Monitor}\label{reference-monitor}}

\begin{itemize}
\tightlist
\item
  classical access enforcement mechanism
\item
  takes request as input
\item
  outputs binary response -\textgreater{} is the request authorized or
  not?
\item
  main components?

  \begin{itemize}
  \tightlist
  \item
    interface
  \item
    authorization module
  \item
    policy store
  \end{itemize}
\end{itemize}

\hypertarget{reference-monitor-interface}{%
\paragraph{Reference Monitor
Interface}\label{reference-monitor-interface}}

\begin{itemize}
\tightlist
\item
  defines queries to the reference monitor
\item
  provides an interface for checking security-sensitive operations

  \begin{itemize}
  \tightlist
  \item
    (security-sensitive means it may violate security policy)
  \end{itemize}
\item
  e.g., consider the \passthrough{\lstinline!open!} system call in UNIX
  (reference monitor decides what is allowed / disallowed)
\end{itemize}

\hypertarget{authorization-module}{%
\paragraph{Authorization Module}\label{authorization-module}}

\begin{itemize}
\tightlist
\item
  takes interface inputs, converts to a query for the policy store
\item
  this query is used to check authorization
\item
  authorization module needs to map PID to subject label and object
  references to an object label
\item
  needs to determine the actual operation(s) to authorize
\end{itemize}

\hypertarget{policy-store}{%
\paragraph{Policy Store}\label{policy-store}}

\begin{itemize}
\tightlist
\item
  database that holds protection state, labeling state, transition state
\item
  answers queries from the authorization module
\item
  has specialized queries for each of the three states
\end{itemize}

\hypertarget{secure-operating-system-definition}{%
\subsubsection{Secure Operating System
Definition}\label{secure-operating-system-definition}}

\begin{itemize}
\tightlist
\item
  a secure operating system's access enforcement satisfies the reference
  monitor model
\item
  the reference monitor model defines the necessary and sufficient
  properties of a system that securely enforces MAC
\item
  three guarantees:

  \begin{enumerate}
  \def\labelenumi{(\arabic{enumi})}
  \tightlist
  \item
    complete mediation -\textgreater{} ensure access enforcement for all
    security-sensitive operations
  \item
    tamper proof -\textgreater{} cannot be tampered with from outside
    the TCB (untrusted processes)
  \item
    verifiable -\textgreater{} small enough to be subject to testing,
    analysis
  \end{enumerate}
\end{itemize}

\hypertarget{assessment-criteria}{%
\subsubsection{Assessment Criteria}\label{assessment-criteria}}

\printbibliography

\end{document}
